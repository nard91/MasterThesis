% Chapter 4  - Analytical Tool

\chapter{Analytical tool} % Main chapter title

\label{Chapter4} % Change X to a consecutive number; for referencing this chapter elsewhere, use \ref{ChapterX}

\lhead{Chapter 4. \emph{Analytical tool}} % Change X to a consecutive number; this is for the header on each page - perhaps a shortened title

%----------------------------------------------------------------------------------------

%DON'T FORGET: reader is an EA and he doesn't know about innovation.

\section{Framework}
%Table with the articles in the paper that refer to each element of the framework
%the framework is the result of a systematic literature review
%Give a review of the results of my search explaining the key ideas and then jump into the framework. Introduce the reader in.

%Framework visual rapresentation of the results + colors

Describe how I’ve generated the framework using different definitions and concepts that are presented in the literature and describe how I have grouped them.

Description of the framework.

\section{Codebook}
%procedure for analyzing the interviews and documents in order to color code the results in the framework
% INTRODUCTION
The codebook is one of the three artifacts produced in this project and plays a crucial role in the analysis of the documents and interviews.
In fact, its use will allow me to scientifically determine if an element of the framework identified by a code is present in the documentation of Cisco or not, and whether Cisco's EA team is aware, contributing or not aware of this element. \\
The codebook follows the structure of the framework and is divided into six parts: innovation as an outcome, innovation as a generation process, innovation as an adoption process, determinants, consequences, and involvement. \\
For all the sections, the examples have been written after the codes have been reviewed and approved by one of the supervisors of the project.
\\ \\ %OUTCOME
In the innovation as an outcome part the labels, codes and examples for each element of the framework related to innovation as an outcome are presented. Each label corresponds to the name an element in the framework and each code the definition of its element revised. In particular, for some of the codes the definitions were referring to innovations and they have been modified to the singular form. \\
In the case of generated radical, generated incremental, adopted incremental and adopted radical innovations I have divided the codes to the basic element (generated, adopted, radical, incremental) to have a better level of detail in the analysis of the documents and interviews.
\\ %PROCESSES
For the sections of the codebook related to the processes of generating and adopting innovations the work has been very similar. In both cases the definition of the process and sources of innovation derives from the framework.
For what concerns the steps of each process they have been labeled with the same nome that they have in framework, but their definitions were not available in the papers that I have taken into consideration. Because of this lack I started looking for the articles cited in Damanpour's article \citep{damanpour2006} and I have been able to find the definitions for each stage of the innovation processes in Rogers book \citep{rogers2003}. \\
As reported in Damanpour's work \citep{damanpour2006}, ``Rogers (1995) presents two innovation processes: (1) innovation development process, which mostly falls under the generation process; and innovation process in organizations, which falls under the adoption process''.
\\ %DETERMINANTS
The section of the codebook related to the determinants there were no definitions in the articles that I have analyzed. This might be because the researchers in that field already agree on common definitions and they have not reported them in their work.
To overcome this shortcoming I have searched on Google for definitions for each element that sounded inline with my understanding of that element. To improve the quality of the code I have listed the key factors synthesized in Crossan's article \citep{crossan2010} that I will use to assess Cisco practices.
In addition to the key factors that decisively affect innovation listed by Crossan I have added the \textit{environmental dynamism} explained by Damanpour \citep{damanpour1998} since I think that this aspect was not fully addressed in Crossan's work. In order to define this element I have synthesized the description provided by the author.
\\ %CONSEQUENCES
In the consequences section I have tried to assign a label that recalled the code that it was referring to. In this section the codes are short since the consequences have been split in small codes.
\\ %INVOLVEMENT LEVEL
The last section of the codebook is the one referring the codes that will be used to check the level of involvement. In fact the EA team can be aware of some elements of the framework, not aware, or they can be contributing and supporting that element of the framework with some tools or activities. Depending on their level of involvement I will color the respective elements in the framework with three colors: \textit{red} for not aware, \textit{yellow} for aware, and \textit{green} for contributing.
\\
Once the framework and the codebook were finished I have designed the questionnaire that guided the interviews. The main reason why the questionnaire artifact has been written as the last one is that it is easier and more accurate to prepare it once all the elements have been identified in the framework and have been defined in the codebook.

\section{Questionnaire}
%since we have applied the funneling technique this is how we have organized the questionnaire

In this section I will describe the questionnaire artifact that structured the phone interviews.

% Funnelling, from open questions to more specific and closed ones
As thoroughly described in section \ref{methQuestionnaire}, the funneling technique has been adopted when designing the framework and as a result each section of the questionnaire starts with very open questions, that allow the interviewed to bring up his thoughts related to that topic, and then when his reply goes out of bounds a list of closed questions have been written to make sure that the data concerning the element of the framework analyzed are gathered. \\

% Questions divided between intro, aware and contribution
% say that the involvement aware --> contribution levels match the flow of open --> close questions that has been adopted.

% Order of questions, from the outcome because is the more understandable and is the key of innovation cite crossan. Then it depends where the interviewed goes, but is planned to be process

% ``Innovation as a process and innovation as an outcome are not equally important. Recall that our definition of innovation includes the aspect of ‘exploitation’. Thus the role of innovation as an outcome is both necessary and sufficient for a successful exploitation of an idea, whereas that of innovation as a process is only necessary but not sufficient. This is why innovation as an outcome is usually the key dependent variable in empirical studies related to innovation''. p. 1169 Crossan \citep{crossan2010}


% Important it has been designed such as that the interviewed has freedom to bring up what is his understanding of innovation and how they are fostering it.


\\ % 		CONCLUDE AND INTRODUCE THE NEXT CHAPTER