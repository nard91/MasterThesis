% Chapter 4  - Analytical Tool

\chapter{Analytical tool} % Main chapter title

\label{Chapter4} % Change X to a consecutive number; for referencing this chapter elsewhere, use \ref{ChapterX}

\lhead{Chapter 4. \emph{Analytical tool}} % Change X to a consecutive number; this is for the header on each page - perhaps a shortened title

%----------------------------------------------------------------------------------------

%DON'T FORGET: reader is an EA and he doesn't know about innovation.

\section{Framework}
%INTRO
In this section I will describe the framework that I have designed as well as the line of though that I have followed. \\
This framework aims to capture the main concepts and theories of innovation that have been presented and explained by the academic community.
Before building the framework I have completed a systematic literature review on innovation as explained in the Methodology section. In table \ref{tab:literaturereview} I have summarized the results of the systematic literature review by listing the main concepts and associating the related papers. These concepts will be explained in the following subsection.

% TABLE THAT SUMMURIZES THE CONCEPT ON INNOVATION
\begin{table}[htbp]
  \centering
  \caption{Innovation concepts and related papers}
    \begin{tabular}{|p{5cm}|p{9cm}|}
    \toprule
    \textbf{Element} & \textbf{Paper} \\
    \midrule
    Administrative innovation & Damanpour 1991, Damanpour 1998, Camizon 2004, Crossan 2009  \\
    Technical innovation & Damanpour 1991, Damanpour 1998, Camizon 2004, Crossan 2009, Baregheh 2009  \\
    Product innovation & Damanpour 1991, Camizon 2004, Crossan 2009, Baregheh 2009  \\
    Process innovation & Damanpour 1991, Camizon 2004, Crossan 2009, Baregheh 2009  \\
    Radical innovation & Damanpour 1991, Damanpour 1998, Camizon 2004, Crossan 2009, Damanpour 2006  \\
    Incremental innovation & Damanpour 1991, Damanpour 1998, Camizon 2004, Crossan 2009, Damanpour 2006  \\
    Determinants & Damanpour 1991, Damanpour 1998, Camizon 2004, Crossan 2009, Damanpour 2006  \\
    Source of innovation & Damanpour 1998, Crossan 2009, Damanpour 2006  \\
    Generation process & Damanpour 1998, Camizon 2004, Crossan 2009, Damanpour 2006, Baregheh 2009  \\
    Adoption process & Damanpour 1998, Camizon 2004, Crossan 2009, Damanpour 2006, Baregheh 2009  \\
    Initiation stage & Damanpour 1991, Damanpour 1998, Damanpour 2006  \\
    Implementation stage & Damanpour 1991, Damanpour 1998, Damanpour 2006, Baregheh 2009  \\
    Environmental conditions & Damanpour 1998 \\
    Direction of innovation process  & Crossan 2009  \\
    Nature of innovation process & Crossan 2009  \\
    Level of innovation process & Crossan 2009  \\
    Driver of innovation process & Crossan 2009, Damanpour 2006  \\
    Service & Crossan 2009, Baregheh 2009  \\
    Business Model & Crossan 2009  \\
    \bottomrule
    \end{tabular}%
  \label{tab:literaturereview}%
\end{table}%

%Give a review of the results of my search explaining the key ideas and then jump into the framework. Introduce the reader in.

%Describe how I’ve generated the framework using different definitions and concepts that are presented in the literature and describe how I have grouped them.
\subsection{Innovation concepts}

%OUTCOME and PROCESS INNOVATION
The academic literature related to innovation presents a fundamental distinction, explained in Crossan's, Damanpour's and Camisón-Zornoza's work, which separates innovation as an outcome and innovation as a process.
In the first case, innovation is intended as a new product, service, process, or technology; while innovation as a process concerns the way in which innovation is created, designed, adopted and commercialized. It will follow a further analysis of these two concepts of innovation. \\

%ADMINISTRATIVE (ADM process and BM) AND TECHNICAL (Product/Service and Technical process) INNOVATIONS - DUAL-CORE THEORY
In the context of innovation as an outcome, the dual-core theory first introduced by Daft in 1978 \citep{damanpour1998} helps distinguishing innovations concerning the social systems from the ones related to the technical systems of the organization; in fact, this theory defines differently administrative innovations and technical innovations. \\
%ADMINISTRATIVE
Administrative innovations, most recently explained in \citep{crossan2010}, are innovations that are related to the coordination and control of the firm, the structure and management of the organization, the administrative processes, and human resources. These innovations can take form as a business model or an administrative process. \\
%Business Model
While business model innovation is defined by Crossan as ``how a company creates, sells, and delivers value to its customers, whether it be new to the firm, customer, or industry"; there is no specific definition for administrative process innovations.
%Administrative process
In this case, from Crossan's definition of process innovation I have kept only the fragments of text concerning administrative process innovation which is the introduction of new management approaches and new technology that can be used to improve management processes. \\
%TECHNICAL
On the other hand, technical innovations pertain to products, services and production process technologies. This type of innovation is related to the primary work activity of the organization and can take form as a product, service or production process \citep{damanpour2006}. \\
%Product/Service
As a product or service innovation is the novelty and meaningfulness of new products introduced to the market in a timely fashion \citep{crossan2010}.
%Technical process
For the definition of production process (also called technical process) I have removed the fragment of text concerning administrative processes from Crossan's definition of process innovation. As a result, technical innovation is the introduction of new production methods and new technology that can be used to improve production processes.

Description of how it will be used: the codes will light up the different sections of the framework.

%Framework visual rapresentation of the results + colors

\section{Codebook}
%procedure for analyzing the interviews and documents in order to color code the results in the framework

% INTRODUCTION
The codebook is one of the three artifacts produced in this project and plays a crucial role in the analysis of the documents and interviews.
In fact, its use will allow me to scientifically determine if an element of the framework identified by a code is present in the documentation of Cisco or not, and whether Cisco's EA team is aware, contributing or not aware of this element. \\
The codebook follows the structure of the framework and is divided into six parts: innovation as an outcome, innovation as a generation process, innovation as an adoption process, determinants, consequences, and involvement. \\
For all the sections, the examples have been written after the codes have been reviewed and approved by one of the supervisors of the project.
\\ \\ %OUTCOME
In the innovation as an outcome part the labels, codes and examples for each element of the framework related to innovation as an outcome are presented. Each label corresponds to the name an element in the framework and each code the definition of its element revised. In particular, for some of the codes the definitions were referring to innovations and they have been modified to the singular form. \\
In the case of generated radical, generated incremental, adopted incremental and adopted radical innovations I have divided the codes to the basic element (generated, adopted, radical, incremental) to have a better level of detail in the analysis of the documents and interviews.
\\ %PROCESSES
For the sections of the codebook related to the processes of generating and adopting innovations the work has been very similar. In both cases the definitions of the process and sources of innovation derive from the framework.
For what concerns the steps of each process the definitions were not available in the papers that I have taken into consideration. Because of this lack I started looking for the articles cited in Damanpour's article \citep{damanpour2006} and I have been able to find the definitions for each stage of the innovation processes in Rogers book \citep{rogers2003}. \\
As reported in Damanpour's work \citep{damanpour2006}, ``Rogers (1995) presents two innovation processes: (1) innovation development process, which mostly falls under the generation process; and innovation process in organizations, which falls under the adoption process''.
In order to use the definition of each step of the generation and adoption of innovation I have clarified the definition of some steps because they were unclear and their use would have resulted in an inconsistent analysis. In addition, the labels of some steps of these processes have been renamed for clearance purposes.
\\ %DETERMINANTS
The section of the codebook related to the determinants there were no definitions in the articles that I have analyzed. This might be because the researchers in that field already agree on common definitions and they have not reported them in their work.
To overcome this shortcoming I have searched on Google for definitions for each element that sounded inline with my understanding of that element. To improve the quality of the code I have listed the key factors synthesized in Crossan's article \citep{crossan2010} that I will use to assess Cisco practices.
In addition to the key factors that decisively affect innovation listed by Crossan I have added the \textit{environmental dynamism} explained by Damanpour \citep{damanpour1998} since I think that this aspect was not fully addressed in Crossan's work. In order to define this element I have synthesized the description provided by the author.
\\ %CONSEQUENCES
In the consequences section I have tried to assign a label that recalled the code that it was referring to. In this section the codes are short since the consequences have been split in small codes.
\\ %INVOLVEMENT LEVEL
The last section of the codebook is the one referring the codes that will be used to check the level of involvement. In fact the EA team can be aware of some elements of the framework, not aware, or they can be contributing and supporting that element of the framework with some tools or activities. Depending on their level of involvement I will color the respective elements in the framework with three colors: \textit{red} for not aware, \textit{yellow} for aware, and \textit{green} for contributing.
\\
Once the framework and the codebook were finished I have designed the questionnaire that guided the interviews. The main reason why the questionnaire artifact has been written as the last one is that it is easier and more accurate to prepare it once all the elements have been identified in the framework and have been defined in the codebook.

\section{Questionnaire}
%since we have applied the funneling technique this is how we have organized the questionnaire

In this section I will describe the questionnaire artifact that structured the phone interviews.

% Funnelling, from open questions to more specific and closed ones
As thoroughly described in section \ref{methQuestionnaire}, the funneling technique has been adopted when designing the framework and as a result each section of the questionnaire starts with very open questions, that allow the interviewed to bring up his thoughts related to that topic, and then when his reply goes out of bounds a list of closed questions have been written to make sure that the data concerning the element of the framework analyzed are gathered.

% Questions divided between intro, aware and contribution
% say that the involvement aware --> contribution levels match the flow of open --> close questions that has been adopted.

% Order of questions, from the outcome because is the more understandable and is the key of innovation cite crossan. Then it depends where the interviewed goes, but is planned to be process

% ``Innovation as a process and innovation as an outcome are not equally important. Recall that our definition of innovation includes the aspect of ‘exploitation’. Thus the role of innovation as an outcome is both necessary and sufficient for a successful exploitation of an idea, whereas that of innovation as a process is only necessary but not sufficient. This is why innovation as an outcome is usually the key dependent variable in empirical studies related to innovation''. p. 1169 Crossan \citep{crossan2010}


% Important it has been designed such as that the interviewed has freedom to bring up what is his understanding of innovation and how they are fostering it.


 % 		CONCLUDE AND INTRODUCE THE NEXT CHAPTER