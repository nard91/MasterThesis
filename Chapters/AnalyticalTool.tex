% Chapter 4  - Analytical Tool

\chapter{Analytical tool} % Main chapter title

\label{Chapter4} % Change X to a consecutive number; for referencing this chapter elsewhere, use \ref{ChapterX}

\lhead{Chapter 4. \emph{Analytical tool}} % Change X to a consecutive number; this is for the header on each page - perhaps a shortened title

%----------------------------------------------------------------------------------------

%DON'T FORGET: reader is an EA and he doesn't know about innovation.

\section{Framework}
%Table with the articles in the paper that refer to each element of the framework
%the framework is the result of a systematic literature review
%Give a review of the results of my search explaining the key ideas and then jump into the framework. Introduce the reader in.

%Framework visual rapresentation of the results

Describe how I’ve generated the framework using different definitions and concepts that are presented in the literature and describe how I have grouped them.

Description of the framework.

\section{Questionnaire}
%since we have applied the funneling technique this is how we have organized the questionnaire
Description of the questionnaire.

\section{Codebook}
%procedure for analyzing the interviews and documents in order to color code the results in the framework
% INTRODUCTION
The codebook is one of the three artifacts produced in this project and plays a crucial role in the analysis of the documents and interviews.
In fact, its use will allow me to scientifically determine if an element of the framework identified by a code is present in the documentation of Cisco or not, and whether Cisco's EA team is aware, contributing or not aware of this element. \\
The codebook follows the structure of the framework and is divided into six parts: innovation as an outcome, innovation as a generation process, innovation as an adoption process, determinants, consequences, and involvement. \\
For all the sections, the examples have been written after the codes have been reviewed and approved by one of the supervisor of the project.
\\ \\ %OUTCOME
In the innovation as an outcome part the labels, codes and examples for each element of the framework related to innovation as an outcome are presented. Each label corresponds to the name an element in the framework and each code the definition of its element revised. In particular, for some of the codes the definitions were referring to innovations and they have been modified to the singular form.
\\ %PROCESSES
For the sections of the codebook related to the processes of generating and adopting innovations the work has been very similar. In both cases the definition of the process and sources of innovation derives from the framework.
For what concerns the steps of each process they have been labeled with the same nome that they have in framework, but their definitions were not available in the papers that I have taken into consideration. Because of this lack I started looking for the articles cited in Damanpour's article \citep{damanpour2006} and there I have been able to find the definitions for each stage of the generating and adoption process. %VERIFY THIS!!!!
\\ %DETERMINANTS
The section of the codebook related to the determinants there were no definitions in the articles that I have analyzed. This might be because the researchers in that field already agree on a common definition and they have not reported it in their work.
To overcome this shortcoming I have searched on Google for definitions for each element that sounded inline with my understanding of that element. To improve the quality of the code I have listed the key factors synthesized in Crossan's article \citep{crossan2010} that I will use to assess Cisco practices.
In addition to the key factors that decisively affect innovation listed by Crossan I have added the \textit{environmental dynamism} explained by Damanpour \citep{damanpour1998} since I think that this aspect was not fully addressed in Crossan's work. In order to define this element I have synthesized the description provided by the author.
\\ %CONSEQUENCES
In the consequences section I have divided the different labels
\\ %INVOLVEMENT LEVEL
%Color coding