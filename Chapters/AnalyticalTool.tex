% Chapter 4  - Analytical Tool

\chapter{Analytical tool} % Main chapter title

\label{Chapter4} % Change X to a consecutive number; for referencing this chapter elsewhere, use \ref{ChapterX}

\lhead{Chapter 4. \emph{Analytical tool}} % Change X to a consecutive number; this is for the header on each page - perhaps a shortened title

%----------------------------------------------------------------------------------------
%	SECTION 1
%----------------------------------------------------------------------------------------

\section{Framework}

Describe how I’ve generated the framework using different definitions and concepts that are presented in the literature and describe how I have grouped them.

Description of the framework.

\section{Questionnaire}
Description of the questionnaire.

\section{Codebook}
The codebook is one of the three artifacts produced in this project and plays a crucial role in the analysis of the documents and interviews.
In fact, its use will allow me to scientifically determine if an element of the framework identified by a code is present in the documentation of Cisco or not, and whether Cisco's EA team is aware, contributing or not aware of this element. \\
The codebook follows the structure of the framework and is divided into six parts: innovation as an outcome, innovation as a generation process, innovation as an adoption process, determinants, consequences, and involvement. \\

In the innovation as an outcome part the labels, codes and examples for each element of the framework related to innovation as an outcome are presented. Each label corresponds to the name an element in the framework and each code the definition of its element revised. In particular, for some of the codes the definitions were referring to innovations and they have been modified to the singular form. \\
The examples have been written after the codes have been reviewed and approved by one of the supervisor of the project. \\


%-----------------------------------
%	SUBSECTION 2
%-----------------------------------
%
%\subsection{Subsection 2}
%Morbi rutrum odio eget arcu adipiscing sodales. Aenean et purus a est pulvinar pellentesque. Cras in elit neque, quis varius elit. Phasellus fringilla, nibh eu tempus venenatis, dolor elit posuere quam, quis adipiscing urna leo nec orci. Sed nec nulla auctor odio aliquet consequat. Ut nec nulla in ante ullamcorper aliquam at sed dolor. Phasellus fermentum magna in augue gravida cursus. Cras sed pretium lorem. Pellentesque eget ornare odio. Proin accumsan, massa viverra cursus pharetra, ipsum nisi lobortis velit, a malesuada dolor lorem eu neque.
%
%----------------------------------------------------------------------------------------
%	SECTION 2
%----------------------------------------------------------------------------------------
%
%\section{Main Section 2}
%
%Sed ullamcorper quam eu nisl interdum at interdum enim egestas. Aliquam placerat justo sed lectus lobortis ut porta nisl porttitor. Vestibulum mi dolor, lacinia molestie gravida at, tempus vitae ligula. Donec eget quam sapien, in viverra eros. Donec pellentesque justo a massa fringilla non vestibulum metus vestibulum. Vestibulum in orci quis felis tempor lacinia. Vivamus ornare ultrices facilisis. Ut hendrerit volutpat vulputate. Morbi condimentum venenatis augue, id porta ipsum vulputate in. Curabitur luctus tempus justo. Vestibulum risus lectus, adipiscing nec condimentum quis, condimentum nec nisl. Aliquam dictum sagittis velit sed iaculis. Morbi tristique augue sit amet nulla pulvinar id facilisis ligula mollis. Nam elit libero, tincidunt ut aliquam at, molestie in quam. Aenean rhoncus vehicula hendrerit.