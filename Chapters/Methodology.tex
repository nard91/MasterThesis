% Chapter 3  - Methodology

\chapter{Methodology} % Main chapter title

\label{Chapter3} % Change X to a consecutive number; for referencing this chapter elsewhere, use \ref{ChapterX}

\lhead{Chapter 3. \emph{Methodology}} % Change X to a consecutive number; this is for the header on each page - perhaps a shortened title

%----------------------------------------------------------------------------------------

\section{Design Science}
%Design science research paradigm
%Explain what it is
%Explain why is relevant given my research question

% Organize the subsection of the methodology with one subsection per step in the design science approach, including how I am addressing that step, what literature I am using to address it.

%Connect the different steps of the project to the design science research strategy

Design science research methodology that it has been used to create the assessment tool.

\subsection{Construction}
\label{subsec:Construction}
How it has been done a systematic literature review of reviews on innovation. %reference the systematic literature review method: research questions while doing the systematic literature review. Explain that why we considered what we have considered and why the rest has not be taken into account.
Search words in the tile, abstract and keywords: \textit{review innovation, meta-review innovation}. Databases: \textit{Scopus, Inspect and Compendex, Science Direct}
Synthesis of the key concepts
Design of framework: grouping of the common theories, identify definitions for each element of the framework that are coherent with the rest of the literature.
Construction of questionnaire. Funneling technique.
\subsubsection*{Codebook}
One of the key components of the tool that I have designed is the codebook. This artifact is a set of codes, definitions, and examples used as a guide to help analyze interview data. It is essential to analyzing qualitative research because it provides a formalized operationalization of the codes \citep{CodebookGuide}. \\
Following the processes and suggestions written by DeCuir-Gunby et al. \citep{CodebookGuide} and MacQueen et al. \citep{CodebookGuide2} I have constructed the codebook used in this project.
I have developed only theory-driven codes since in this project I am identifying and synthesizing concepts from the theory and checking their presence in a real world environment. As explained in one of the guides cited above \citep{CodebookGuide}, when creating theory-driven codes three steps are generally executed: (1) generate the code, (2) review and revise the code in context of the data, and (3) determine the reliability of the code. \\

Generating a first version of the codes has been a straight forward process. In fact, I have assigned as labels of the code the name of the that element in the framework (e.g. Administrative innovation) and as definition I have used the one in the framework. Only for the codes in the determinant section I had to find a definition of that element since neither the framework and articles had it. Most of the time I have found a clear concise definition on the internet and I've used that one.

%Review and revision of the codes in the context of the data. It will be done once I have received Gustav documents.

I haven't determined the reliability of the codes using a statistical technique since all the practices explained in the cited papers require multiple coders and I have been working alone in this project. Though the codes reliability and consistency has been check in a series of meetings with James Lapalme.

\subsection{Evaluation}
Evaluation of the tool.

%----------------------------------------------------------------------------------------

% READER must understand:
%the student has justified why is the research paradigm correct
%the student knows how to excute correctly that design strategy
%the student has used the various methods in the execution adequately and he has justified why he has used those
