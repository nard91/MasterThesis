% Chapter 3  - Methodology

\chapter{Methodology} % Main chapter title

\label{Chapter3} % Change X to a consecutive number; for referencing this chapter elsewhere, use \ref{ChapterX}

\lhead{Chapter 3. \emph{Methodology}} % Change X to a consecutive number; this is for the header on each page - perhaps a shortened title

%----------------------------------------------------------------------------------------
%	SECTION 1
%----------------------------------------------------------------------------------------

\section{Design Science}
Design science research methodology that it has been used to create the assessment tool.

\subsection{Construction}
\label{subsec:Construction}
How it has been done the literature review of reviews on innovation. Search words in the tile, abstract and keywords: \textit{review innovation, meta-review innovation}. Databases: \textit{Scopus, Inspect and Compendex, Science Direct}
Synthesis of the key concepts
Design of framework: grouping of the common theories, identify definitions for each element of the framework that are coherent with the rest of the literature.
Construction of questionnaire. Funneling approach.
\subsubsection*{Codebook}
One of the key components of the tool that I have designed is the codebook. This artifact is a set of codes, definitions, and examples used as a guide to help analyze interview data. It is essential to analyzing qualitative research because it provides a formalized operationalization of the codes \citep{CodebookGuide}. \\
Following the processes and suggestions written by DeCuir-Gunby et al. \citep{CodebookGuide} and MacQueen et al. \citep{CodebookGuide2} I have constructed the codebook used in this project.
I have developed only theory-driven codes since in this project I am identifying and synthesizing concepts from the theory and checking their presence in a real world environment. As explained in one of the guides cited above \citep{CodebookGuide}, when creating theory-driven codes three steps are generally executed: (1) generate the code, (2) review and revise the code in context of the data, and (3) determine the reliability of the code. \\

Generating a first version of the codes has been a straight forward process. In fact, I have assigned as labels of the code the name of the that element in the framework (e.g. Administrative innovation) and as definition I have used the one in the framework. Only for the codes in the determinant section I had to find a definition of that element since neither the framework and articles had it. Most of the time I have found a clear concise definition on the internet and I've used that one.

%Review and revision of the codes in the context of the data. It will be done once I have received Gustav documents.

I haven't determined the reliability of the codes using a statistical technique since all the practices explained in the cited papers require multiple coders and I have been working alone in this project. Though the codes reliability and consistency has been check in a series of meetings with James Lapalme.

\subsection{Evaluation}
Evaluation of the tool.

%-----------------------------------
%	SUBSECTION 1
%-----------------------------------
%\subsection{Subsection 1}
%
%Nunc posuere quam at lectus tristique eu ultrices augue venenatis. Vestibulum ante ipsum primis in faucibus orci luctus et ultrices posuere cubilia Curae; Aliquam erat volutpat. Vivamus sodales tortor eget quam adipiscing in vulputate ante ullamcorper. Sed eros ante, lacinia et sollicitudin et, aliquam sit amet augue. In hac habitasse platea dictumst.
%
%-----------------------------------
%	SUBSECTION 2
%-----------------------------------
%
%\subsection{Subsection 2}
%Morbi rutrum odio eget arcu adipiscing sodales. Aenean et purus a est pulvinar pellentesque. Cras in elit neque, quis varius elit. Phasellus fringilla, nibh eu tempus venenatis, dolor elit posuere quam, quis adipiscing urna leo nec orci. Sed nec nulla auctor odio aliquet consequat. Ut nec nulla in ante ullamcorper aliquam at sed dolor. Phasellus fermentum magna in augue gravida cursus. Cras sed pretium lorem. Pellentesque eget ornare odio. Proin accumsan, massa viverra cursus pharetra, ipsum nisi lobortis velit, a malesuada dolor lorem eu neque.
%
%----------------------------------------------------------------------------------------
%	SECTION 2
%----------------------------------------------------------------------------------------
%
%\section{Main Section 2}
%
%Sed ullamcorper quam eu nisl interdum at interdum enim egestas. Aliquam placerat justo sed lectus lobortis ut porta nisl porttitor. Vestibulum mi dolor, lacinia molestie gravida at, tempus vitae ligula. Donec eget quam sapien, in viverra eros. Donec pellentesque justo a massa fringilla non vestibulum metus vestibulum. Vestibulum in orci quis felis tempor lacinia. Vivamus ornare ultrices facilisis. Ut hendrerit volutpat vulputate. Morbi condimentum venenatis augue, id porta ipsum vulputate in. Curabitur luctus tempus justo. Vestibulum risus lectus, adipiscing nec condimentum quis, condimentum nec nisl. Aliquam dictum sagittis velit sed iaculis. Morbi tristique augue sit amet nulla pulvinar id facilisis ligula mollis. Nam elit libero, tincidunt ut aliquam at, molestie in quam. Aenean rhoncus vehicula hendrerit.