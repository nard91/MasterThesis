% Chapter 2  - Literature Review

\chapter{Literature Review} % Main chapter title

\label{Chapter2} % Change X to a consecutive number; for referencing this chapter elsewhere, use \ref{ChapterX}

\lhead{Chapter 2. \emph{Literature Review}} % Change X to a consecutive number; this is for the header on each page - perhaps a shortened title

%----------------------------------------------------------------------------------------

\section{EA research}
%Position the paper related to EA work.


%2. What progress has been made since these seminal works?
%3. What are the most relevant recent works? What is the best order to mention these works?
%4. What are the achievements and limitations of these recent works?
%5. What gap do these limitations reveal?
%6. How does my work intend to fill this gap?

%1. What are the seminal works on my topic? Do I need to mention these? James's work

Explain the context of the project, the three schools of thought.

In the field of EA three schools of though have been identified by LaPalme in \citep{lapalme2012} each with its own scope and purpose. In this section each school of though will be presented with its definition, objectives, principles, challenges and limitations. \\
%introduction of the scope and purspose of each school and name of the schools of though.
From the literature emerge three scopes and purposes of EA. The first one, that has been named by professor Lapalme \textit{Enterprise IT Architecting}, refers to the term 'enterprise' as the enterprise-wide IT platform, including all components of the enterprise IT assets. In this perception of EA the purpose is to effectively execute and operate the overall enterprise strategy for maintaining a competitive advantage by aligning the business and IT strategies such that the proper IT capabilities are developed to support current and future business needs. \\
\textit{Enterprise Intergrating} is the second school of though and in this case the enterprise is defined as a sociocultural, techno-economic system including all facets of the enterprise. EA's purpose is to effectively implement the overall enterprise strategy by designing the various enterprise facets (governance structure, IT capabilities, remuneration policies, work design, etc.) to maximize coherency between them and minimize contradictions. \\
The last school of though, \textit{Enterprise Ecological Adaptation}, conceptualizes the enterprise in its environment with its bidirectional relationship and transactions between the enterprise and its environment. And help the organization to innovate and adapt by designing the various enterprise facets to maximize organizational learning throughout the enterprise is its purpose. \\

\subsubsection*{Enterprise IT Architecting}

\subsubsection*{Enterprise Integrating}

\subsubsection*{Enterprise Ecological Adaptation}

%Three schools of thought have been proposed however currently there are few tool which are relevant for working in the third school and this is where I am contributing.
% Clear link on what I'm contributing

\section{Innovation research}
Position the paper related to innovation work, measurement of innovation Crossan.

%1. What are the seminal works on my topic? Do I need to mention these?
%2. What progress has been made since these seminal works?
%3. What are the most relevant recent works? What is the best order to mention these works?
%4. What are the achievements and limitations of these recent works?
%5. What gap do these limitations reveal?
%6. How does my work intend to fill this gap?




\section{EA tools for innovation research}
Intro on the tools related to EA, maturity and others.
%Is it tool the right name for our artifact
%quick search on the literature with EA and innovation tools.
%Go and look into togaf and togaf enumerates a number of variuos methodologies and techniques

Position the paper in the context of EA tools for assess innovation.


\section{Constribution}
Restate with more detail what my contribution is.

Contribution - Synthesis and integration of the literature on innovation at an adequate level of detail for applying innovation knowledge in the context of EA. 
Synthesis of the key concepts of innovation. 
Creation of an assessment tool: framework, questionnaire and codebook. 
Test of the tool in a real world environment.

%1. What are the seminal works on my topic? Do I need to mention these?
%2. What progress has been made since these seminal works?
%3. What are the most relevant recent works? What is the best order to mention these works?
%4. What are the achievements and limitations of these recent works?
%5. What gap do these limitations reveal?
%6. How does my work intend to fill this gap?

%Innovation has been at the center of academic research for many decades. As a result, past research on innovation in organizations has examined its determinants, processes, and consequences \citep{damanpour2006}. The theoretical base for innovation, though, is still fragmented and incomplete \citep{crossan2010}.
%In the literature, most of the definitions that have been given to innovation are discipline oriented and, as presented in \citep{crossan2010}, there is a lack of consensual definition of innovation. In recent years several researchers tried to overcome this problem by developing a universal definition of innovation. Baregheh in \citep{baregheh2009} presents an analysis of the definitions of innovation in several fields (economics, entrepreneurship, business and management, and technology) and suggests a general and integrative definition of organizational innovation. Crossan in \citep{crossan2010}, through a systematic review of the literature on innovation of the past 27 years, consolidates the large body of knowledge on innovation into a framework of organizational innovation. During several years, Damanpour et al. elaborated new theories in order to consolidate the inconsistent results of innovation research. In this paper we have analyzed Damanpour’s work related to the determinants and moderators of organizational innovation \citep{damanpour1991}, the role of the environmental change and how it affected organizational structure \citep{damanpour1998}, and the distinction between generation and adoption of innovation \citep{damanpour2006}.
%
%Related to innovation in organizations, [6] describes that innovations can be either generated or adopted. “Innovations that are generated in an organization are usually for its own use or for sale to other organizations and the generation of innovation is a process which results in an outcome - a new product, service, program, or technology. If this outcome in then acquired by another organization, the second organization goes through another process, that of adopting the innovation. [...] The generation process results in innovation as an outcome for the generation organization, while the adoption process is responsible for how that innovation is assimilated in the adopting organization” [6].
%
%The environment is recognized as one of the important contextual factors that influences innovation, and environmental change is often seen as a driving force for organizational innovation [6]. Also organizational determinants have been taken into consideration since Damanpour proved their relation with innovation [5].
%
