% Chapter 2  - Literature Review

\chapter{Literature Review} % Main chapter title

\label{Chapter2} % Change X to a consecutive number; for referencing this chapter elsewhere, use \ref{ChapterX}

\lhead{Chapter 2. \emph{Literature Review}} % Change X to a consecutive number; this is for the header on each page - perhaps a shortened title

%----------------------------------------------------------------------------------------

\section{EA research}
%Position the paper related to EA work.

In the field of EA three schools of though have been identified by LaPalme in \citep{lapalme2012} each with its own scope and purpose. In this section each school of though will be presented with its definition, objectives, principles, challenges and limitations. \\
%introduction of the scope and purspose of each school and name of the schools of though.
From the literature emerge three scopes and purposes of EA. The first one, that has been named by professor Lapalme \textit{Enterprise IT Architecting}, refers to the term 'enterprise' as the enterprise-wide IT platform, including all components of the enterprise IT assets. In this perception of EA the purpose is to effectively execute and operate the overall enterprise strategy for maintaining a competitive advantage by aligning the business and IT strategies such that the proper IT capabilities are developed to support current and future business needs. \\
\textit{Enterprise Intergrating} is the second school of though and in this case the enterprise is defined as a sociocultural, techno-economic system including all facets of the enterprise. EA's purpose is to effectively implement the overall enterprise strategy by designing the various enterprise facets (governance structure, IT capabilities, remuneration policies, work design, etc.) to maximize coherency between them and minimize contradictions. \\
The last school of though, \textit{Enterprise Ecological Adaptation}, conceptualizes the enterprise in its environment with its bidirectional relationship and transactions between the enterprise and its environment. And help the organization to innovate and adapt by designing the various enterprise facets to maximize organizational learning throughout the enterprise is its purpose. \\

\subsubsection*{Enterprise IT Architecting}

\subsubsection*{Enterprise Integrating}

\subsubsection*{Enterprise Ecological Adaptation}

%Three schools of thought have been proposed however currently there are few tool which are relevant for working in the third school and this is where I am contributing.
% Clear link on what I'm contributing

\section{Innovation research}
Position the paper related to innovation work, measurement of innovation Crossan.

\section{EA tools for innovation research}
Intro on the tools related to EA, maturity and others.
%Is it tool the right name for our artifact
%quick search on the literature with EA and innovation tools.
%Go and look into togaf and togaf enumerates a number of variuos methodologies and techniques

Position the paper in the context of EA tools for assess innovation.


\section{Constribution}
Restate with more detail what my contribution is.

Contribution - Synthesis and integration of the literature on innovation at an adequate level of detail for applying innovation knowledge in the context of EA. 
Synthesis of the key concepts of innovation. 
Creation of an assessment tool: framework, questionnaire and codebook. 
Test of the tool in a real world environment.