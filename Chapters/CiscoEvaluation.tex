% Chapter 5  - Evaluation at Cisco

\chapter{Evaluation at Cisco} % Main chapter title

\label{Chapter5} % Change X to a consecutive number; for referencing this chapter elsewhere, use \ref{ChapterX}

\lhead{Chapter 5. \emph{Evaluation at Cisco}} % Change X to a consecutive number; this is for the header on each page - perhaps a shortened title

%----------------------------------------------------------------------------------------
%	SECTION 1
%----------------------------------------------------------------------------------------

\section{EA at Cisco}

In this chapter I will succinctly introduce Cisco and their EA department, and the employees that I have interviewed.
%-----------------------------------
%	Cisco
%-----------------------------------

%Line of business, strategy, markets.

%-----------------------------------
%	EA Department
%-----------------------------------
\subsubsection*{Cisco's EA department}
%STRUCTURE
Cisco's EA department is divided in: 160 enterprise architects roles (100 people), and they are purely strategic focused; and 160 solution designer roles (100 people), and they are solution architects.
There are three communities of architects at Cisco: Business Architects, System Architects and Technology Architects.
%MOTIVATION
 the motivation that Cisco has in pursuing EA practices. Gustav told me that EA allows them the recognize opportunities and organize them selves in a way that they can exploit these opportunities.
Doing EA has allowed Cisco to capture opportunities and plan shared and collective initiatives and investments.
%PAST
In the past IT has been responsible to solve lot of the problems existing in the business environment and EA was focused on technical architecture. Cisco's EA team has modeled current and future states of the organization trying to provide solutions to the business challenges introducing changes in the operations of the organization.
%PRESENT
EA now is moving towards an orchestrated EA in which architects both operations \& business, and systems \& technologies. Business and system architects work together to find a way to exploit the opportunities that have been already recognized.
Based on their decisions, investments are made in all the views (BOST) of the organization.
Cisco's EA team is doing this following the capability-based planning approach. The enterprise architects work on the strategic level and there are solution architects that design the specific solutions.

%-----------------------------------
%	Employees
%-----------------------------------
\subsubsection*{Interviewees}

Gustav Normark Toppenberg - Sr Manager - Enterprise Architecture

Richard Hare - Enterprise Architect at Cisco Systems - Senior Director of Business Architecture

%-----------------------------------
%	RESULTS
%-----------------------------------

\section{Results of the assessment tool}
Describe the results.
%we looked for evidence on this concept and it wasn't there, or these are the evidence that we have found+ example of what was said.

In this section I will present the results of the coding activity in three parts related to documents, interviews, and the combination of the two.

%-----------------------------------
%	Documents
%-----------------------------------
\subsection{Documents}

I have started assessing Cisco's EA team from a series of documents that they shared with me.

\subsubsection*{Architecture Practice - Architecture Led Investment Planning}
The operational play book covers the Change the Business planning portion of Cisco's Architecture Led Planning process. Among other things, this process enables Change the Business (CtB) prioritization based on a cross-functional architecture view and align ongoing Architecture Planning with Fiscal Year Planning activities.
Analyzing the document \citep{architecturePractice} four matches with the codebook elements have been identified:

\begin{enumerate}

\item ``Each IT Architecture Grouping collects new Change the Business (CtB) opportunities from their stakeholders''
\item ``Each IT Architecture Grouping assesses their Change the Business Opportunities and leverages the BOST Reference Model to define the architecture dependencies and relationships across the Business, Operational, Systems, and Technology views'' 
\item ``Each Architecture Grouping updates their existing Architecture Roadmaps to reflect the new Systems and Technology Capabilities that are required to support the new Change the Business Opportunities''
\item ``Each Architecture Grouping structures the Target Systems and Technology Architecture Capabilities scope into a Program and Project hierarchy''

\end{enumerate}

The first one is a match of the \textit{adoption of innovation agenda-setting stage} code because it refers to the identification of new business opportunities and this is the first step of the adoption process, the identification and prioritization of new opportunities. In addition, since is the IT architecture team performing the task of collecting new business opportunities Cisco's EA practices are \textit{supporting} Cisco's innovation.
\\
The other three fragments refer to the \textit{adoption of innovation redefining and restructuring stage} code because they all relate to activities that involve the adaptation of the organization to accommodate the new business opportunity. As for the previous fragment, teams of Cisco's EA department are performing these activities so also in this case they are \textit{supporting} Cisco's innovation.

%-----------------------------------
%	Interviews
%-----------------------------------
\subsection{Interviews}

%-----------------------------------
%	Summary
%-----------------------------------
\subsection{Combination}


%If the first two do not match, in the results we indicated that ..... with reserve becuase we have found evidence only from one source, weak evidence
